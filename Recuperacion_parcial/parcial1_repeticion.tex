\documentclass[12pt,letterpaper]{article}

\usepackage[latin1]{inputenc}
\usepackage[spanish]{babel}
\usepackage{amsmath}
\usepackage{amsfonts}
\usepackage{amssymb}
\usepackage{graphicx}
\usepackage{fancyhdr} % Required for custom headers
\usepackage{lastpage} % Required to determine the last page for the footer
\usepackage{extramarks} % Required for headers and footers
\usepackage[usenames,dvipsnames]{color} % Required for custom colors
\usepackage{listings} % Required for insertion of code
\usepackage{courier} % Required for the courier font
\usepackage{multirow}
\usepackage{hyperref}
\usepackage[left=4cm,right=2cm,top=2cm,bottom=2cm]{geometry}

% Margins
\topmargin=-0.45in
\evensidemargin=0in
\oddsidemargin=0in
\textwidth=6.5in
\textheight=9.0in
\headsep=0.25in

\linespread{1.1} % Line spacing
\newcommand{\horrule}[1]{\rule{\linewidth}{#1}}

% Creates a new command to include a perl script, the first parameter is the filename of the script (without .pl), the second parameter is the caption
\newcommand{\perlscript}[2]{
\begin{itemize}
\item[]\lstinputlisting[caption=#2,label=#1]{#1.cs}
\end{itemize}
}

\author{Santiago V�squez}

\begin{document}

\begin{tabular}{l l}
\multirow{5}{*}{\includegraphics[width=2cm]{../recursos/logo.png}}
 & Universidad del Istmo de Guatemala \\
 & Facultad de Ingenier�a \\
 & Inform�tica 1 \\
 & Henry Santiago V�squez Alvizures \\
 & ID: 20181106 \\
\end{tabular}

\begin{center}
	\horrule{0.5pt}
	\huge{Repetici�n Examen Parcial \#1} \\
	\large{Fecha: 14 de agosto, 2018 } \\
    \horrule{1pt}
\end{center}

% \perlscript{homework_example}{Sample Perl Script With Highlighting}

\section*{Problema \#1: Definir el conjunto de nodos }
\begin{enumerate}
	\item{Tomando las islas como nodos y coloc�ndoles a cada una una etiqueta con A, B, C o D, podemos definir entonces el conjunto:} \\
	{\bf Respuesta:} $\{A, B , C, D\}$
	
 	\item{El conjunto de v�rtices del grafo} \\
	{\bf Respuesta:} \\
	\begin{tabular}{p{6cm} c}
    $$
     \left\{
        \begin{bmatrix}
            \langle A,B \rangle & \langle A,D \rangle & \langle B,C \rangle & \langle B,D \rangle \\
            \langle D,A \rangle  \\
        \end{bmatrix}
    \right\} 
	$$\\
	\end{tabular}
\end{enumerate}
\section*{Problema \#2: }
Demostrar utilizando inducci\'on que la formula de Gauss para sumatorias es correcta:
\[
        \sum_{i=1}^{n}{i}=\frac{n(n+1)}{2}       
\]
Por caso base donde n = 1
\[ 
        \frac{1(1+1)}{2} =  1       
\]
Por caso inductivo
\[ 
        \sum_{i=1}^{n}{i}=\frac{n(n+1)}{2} 
\]
Se supone que n = k: 
\[        
        1+2+3+...+k=  \frac{k(k+1)}{2}    
\]
Verificamos que se cumpla para = k + 1: 
\[        
        \sum_{i=1}^{k+1}{i}=\frac{(k+1)((k+1)+1)}{2}       
\]
\[        
        =\frac{(k+1)((k+2)}{2}       
\]
\[        
		\sum_{i=1}^{k+1}{i}= 1+2+3+...+k+k+1       
\]
\[        
		\frac{k(k+1)}{2}  + k +1
\]
\[        
		\frac{k(k+1)+2(k+1)}{2}
\]
\[        
		\frac{(k+1)+(k+2)}{2}
\]
Se cumple entonces que: 
\[ 
        \sum_{i=1}^{n}{i}=\frac{n(n+1)}{2} 
\]

donde $\sum_{i=1}^{n}i=1+2+3+4+\ \ldots\ +n$.
\\\\
Para esta demostraci\'on, su caso base debe ser
$n=1$ en vez de $n=0$. Sin embargo, la demostraci\'on
del caso inductivo procede de la misma forma que
se ha estudiado en clase.
\section*{Problema \#3: }



\section*{Problema \#4: }
Demostrar por medio de inducci\'on la comutatividad de la suma de
numeros naturales unarios: $a\oplus b = b\oplus a$ \\

Inducci�n sobre a:\\
Caso base: Demostrar P(0)= 0 + b = b + 0

\[
		0 + b = b 
\]
\[
		 = b + 0 
\]

Por hip�tesis de inducci�n:\\
Suponemos que P(a)= a + b = b + a\\
Paso Inductivo: Demostrar P(s(a))= s(a) + b = b + s(a)\\
Tomando el lado derecho:
\[
	 b + s(a) = s (a + b)
	 = s (a + b) 
\]

\section*{Problema \#5: }



\end{document}