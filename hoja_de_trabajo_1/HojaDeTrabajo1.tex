\documentclass[12pt,letterpaper]{article}

\usepackage[latin1]{inputenc}
\usepackage[spanish]{babel}
\usepackage{amsmath}
\usepackage{amsfonts}
\usepackage{amssymb}
\usepackage{graphicx}
\usepackage{fancyhdr} % Required for custom headers
\usepackage{lastpage} % Required to determine the last page for the footer
\usepackage{extramarks} % Required for headers and footers
\usepackage[usenames,dvipsnames]{color} % Required for custom colors
\usepackage{listings} % Required for insertion of code
\usepackage{courier} % Required for the courier font
\usepackage{multirow}
\usepackage{hyperref}
\usepackage[left=4cm,right=2cm,top=2cm,bottom=2cm]{geometry}

% Margins
\topmargin=-0.45in
\evensidemargin=0in
\oddsidemargin=0in
\textwidth=6.5in
\textheight=9.0in
\headsep=0.25in

\linespread{1.1} % Line spacing
\newcommand{\horrule}[1]{\rule{\linewidth}{#1}}

% Creates a new command to include a perl script, the first parameter is the filename of the script (without .pl), the second parameter is the caption
\newcommand{\perlscript}[2]{
\begin{itemize}
\item[]\lstinputlisting[caption=#2,label=#1]{#1.cs}
\end{itemize}
}

\author{Santiago V�squez}

\begin{document}

\begin{tabular}{l l}
\multirow{5}{*}{\includegraphics[width=2cm]{../recursos/logo.png}}
 & Universidad del Istmo de Guatemala \\
 & Facultad de Ingenier�a \\
 & Inform�tica 1 \\
 & Henry Santiago V�squez Alvizures \\
 & ID: 20181106 \\
\end{tabular}

\begin{center}
	\horrule{0.5pt}
	\huge{Hoja de Trabajo \#1} \\
	\large{Fecha: 25 de Julio, 2018 } \\
    \horrule{1pt}
\end{center}

% \perlscript{homework_example}{Sample Perl Script With Highlighting}

\section*{Ejercicio \#2: Abstracci�n }
\begin{enumerate}
	\item{El conjunto de nodos del grafo} \\
	{\bf Respuesta:} $\{1, 2, 3, 4, 5, 6\}$
	
 	\item{El conjunto de v�rtices del grafo} \\
	{\bf Respuesta:} \\
	\begin{tabular}{p{6cm} c}
    $$
     \left\{
        \begin{bmatrix}
            \langle 2,1 \rangle & \langle 1,4 \rangle & \langle 6,5 \rangle & \langle 5,1 \rangle \\
            \langle 2,6 \rangle & \langle 1,5 \rangle & \langle 6,2 \rangle & \langle 5,6 \rangle \\
            \langle 2,4 \rangle & \langle 1,2 \rangle & \langle 6,4 \rangle & \langle 5,3 \rangle \\
            \langle 2,3 \rangle & \langle 1,3 \rangle & \langle 6,3 \rangle & \langle 5,4 \rangle \\ 
        \end{bmatrix}
    \right\} 
	$$\\
	\end{tabular}
\end{enumerate}
\section*{Ejercicio \#4: }
\begin{enumerate}
	\item{�Que estructura de datos podria representar un lanzamiento de dados?}\\
		{\bf Respuesta:} Estructura de datos de grafos/caminos.
	\item{�Que algoritmo podriamos utilizar para generar dicha estructura?}\\
		{\bf Respuesta:} Algoritmo de caminos
    \item{�Como nos aseguramos que ese algoritmo siempre produce un resultado?}\\
    	{\bf Respuesta:} Para la determinaci�n del camino m�s corto, dado un v�rtice origen, hacia el resto de los v�rtices en un grafo, se deben explorar todos los caminos m�s cortos que parten del vertice de origen y que llevan a todos los dem�s v�rtices; cuando 						 se obtiene el camino m�s corto desde el v�rtice origen hasta el resto de los 								 v�rtices que componen el grafo, el algoritmo se detiene, logrando asi descartar 							 ciclos. Por ello el Algoritmo de caminos siempre produce un resultado.
\end{enumerate}

\end{document}