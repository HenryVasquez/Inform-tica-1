\documentclass[12pt,letterpaper]{article}

\usepackage[latin1]{inputenc}
\usepackage[spanish]{babel}
\usepackage{amsmath}
\usepackage{amsfonts}
\usepackage{amssymb}
\usepackage{graphicx}
\usepackage{fancyhdr} % Required for custom headers
\usepackage{lastpage} % Required to determine the last page for the footer
\usepackage{extramarks} % Required for headers and footers
\usepackage[usenames,dvipsnames]{color} % Required for custom colors
\usepackage{listings} % Required for insertion of code
\usepackage{courier} % Required for the courier font
\usepackage{multirow}
\usepackage{hyperref}
\usepackage[left=4cm,right=2cm,top=2cm,bottom=2cm]{geometry}

% Margins
\topmargin=-0.45in
\evensidemargin=0in
\oddsidemargin=0in
\textwidth=6.5in
\textheight=9.0in
\headsep=0.25in

\linespread{1.1} % Line spacing
\newcommand{\horrule}[1]{\rule{\linewidth}{#1}}

% Creates a new command to include a perl script, the first parameter is the filename of the script (without .pl), the second parameter is the caption
\newcommand{\perlscript}[2]{
\begin{itemize}
\item[]\lstinputlisting[caption=#2,label=#1]{#1.cs}
\end{itemize}
}

\author{Santiago V�squez}

\begin{document}

\begin{tabular}{l l}
\multirow{5}{*}{\includegraphics[width=2cm]{../recursos/logo.png}}
 & Universidad del Istmo de Guatemala \\
 & Facultad de Ingenier�a \\
 & Inform�tica 1 \\
 & Henry Santiago V�squez Alvizures \\
 & ID: 20181106 \\
\end{tabular}

\begin{center}
	\horrule{0.5pt}
	\huge{Hoja de Trabajo \#2} \\
	\large{Fecha: 02 de Agosto, 2018 } \\
    \horrule{1pt}
\end{center}

\emph{Instrucciones: Resolver cada uno de los ejercicios siguiendo sus respectivas
instrucciones. El trabajo debe ser entregado a traves de Github, en su repositorio del curso, colocado en una carpeta llamada "Hoja de trabajo 2".
Al menos que la pregunta indique diferente, todas las respuestas a preguntas escritas deben presentarse en
un documento formato pdf, el cual haya sido generado mediante Latex. }

% \perlscript{homework_example}{Sample Perl Script With Highlighting}
\section*{Ejercicio \#1 }

\begin{flushleft}
	Demostrar utilizando inducci\'on:
		\[
       		 \forall\ n.\ n^3\geq n^2
		\]
	\\donde $n\in\mathbb{N}$
	\\
	{\bf Respuesta: } \\
	Caso base:
		\[
       		 n = 0, 0^3\geq 0^2
		\]
		\[
       		  0\geq 0
		\]
	Por hip�tesis inductiva:
		\[
       		 (n+1)*(n+1)^2 \geq (n+1)^2
       		 
		\]
		\[
				(n+1)\geq(n+1)^2/(n+1)^2
		\]
		\[
				(n+1)\geq 1
		\]	
		\[
				(n+1)-1\geq 1-1
		\]
		\[
				n \geq 0
		\]		
	
	
	
\end{flushleft}


\section*{Ejercicio \#2}
\begin{flushleft}
	Demostrar utilizando inducci\'on la desigualdad de Bernoulli:
		\[
        	\forall\ n.\ (1+x)^n\geq nx
		\]
		\\donde $n\in \mathbb{N}$, $x\in \mathbb{Q}$ y $x\geq -1$
		\\
		
		
	{\bf Respuesta: } \\
	Caso base:
		\[
       		 n = 0,  (1+x)^n\geq nx\\ pero\ sabemos\ que\ tambi�n\ (1+x)^n\geq nx +1 
		\]
		\[
       		 Entonces,\    (1+x)^0\geq (0)x +1
		\]
		\[
       		    		   1\geq 1
		\]
		Por hip�tesis inductiva:
		\[
       		 Suponesmos\ que\ (1+x)^n\geq nx +1
		\]
		\[
       		 Entonces,\    (1+x)^0\geq (0)x +1
		\]
		\[
        	 x\geq -1\ esto\ implica\ que\ x+1\geq 0
		\]
		\[
       		 (1+x)^n\ (1+x)\geq (1+nx)\ (1+x)
		\]
		\[
				(1+x)^{n+1}\geq{(1+nx)(1+x)}=1+(n+1)x+nx^2
		\]
		\[
				nx^2\geq 0
		\]
		\[
				(1+x)^{n+1}\geq{1+(n+1)x
		\]

La desigualdad se hace evidente para n+1 porque sabemos que $nx^2$ es un n�mero 		positivo.
	
	
		
\end{flushleft}
{\bf Consejo: }Es possible demostrar esto demostrando una propiedad m\'as fuerte
donde el lado izquierdo es mayor que $nx + 1$




\end{document}