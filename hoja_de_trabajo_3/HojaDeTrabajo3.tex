\documentclass[12pt,letterpaper]{article}
\usepackage[utf8]{inputenc}
\usepackage[spanish]{babel}
\usepackage{graphicx}
\usepackage[left=4cm,right=2cm,top=2cm,bottom=2cm]{geometry}
\author{Henry Santiago Vásquez Alvizures}
\title{Hoja de Trabajo No. 3}
\begin{document}
\maketitle 
\section{Ejercicio \#1}
\begin{flushleft}
	$ s(s(s(s(0))))  \oplus s(s(s(0))) $\\
	$ s(s(s(s(s(0)))))  \oplus s(s(0)) $\\
	$ s(s(s(s(s(s(0))))))  \oplus s(0) $\\
	$ s(s(s(s(s(s(s(0 \oplus 0)))))))  $\\
	$ s(s(s(s(s(s(s(0)))))))  $\\
\end{flushleft}
\section{Ejercicio \#2}
\begin{flushleft}
	$ s(i) \oplus (s(i) \otimes j) $\\
\end{flushleft}
\section{Ejercicio \#3}
\begin{itemize}
	\item{$s(s(s(0)))\otimes 0$}\\
	Por definición sabemos que cualquier número multiplicado por 0 es igual a 0.\\
	Entonces: $s(s(s(0)))\otimes 0 = 0$
	
	\item{$s(s(s(0)))\otimes s(0)$}
	\[
		s(s(s(0))) \oplus (s(s(s(0))) \otimes 0)\\	
	\]
	\[
		s(s(s(0))) \oplus (0)\\	
	\]
	\[
		s(s(s(0+0)))\\	
	\]
	\[
		s(s(s(0)))\\	
	\]
	
	\item{$s(s(s(0))) \otimes s(s(0))$}	
	\[
		s(s(s(0))) \oplus (s(s(s(0))) \otimes s(0))
	\]
	\[
		s(s(s(0))) \oplus (s(s(s(0))) \otimes s(0))
	\]
	\[
		s(s(s(0))) \oplus s(s(s(0)))
	\]
	\[
		s(s(s(s(0)))) \oplus s(s(0))
	\]
	\[
		s(s(s(s(s(0))))) \oplus s(0)
	\]
	\[
		s(s(s(s(s(s(0 + 0)))))) 
	\]
	\[
		s(s(s(s(s(s(0)))))) 
	\]
	
\end{itemize}
\section{Ejercicio \#4}
\begin{itemize}
		\item{$a\oplus s(s(0))=s(s(a))$}
		\[
			s(s(a + 0)) = s(s(a))
		\]
		\[
			s(s(a)) = s(s(a))
		\]
		
        \item{$a \otimes b = b \otimes a$}\\
        Caso base: a = 0
        \[
			0 \otimes b = b \otimes 0
        \]
        \[
			0 = 0
        \]
        Hi Inductiva: a = n + 1
        \[
			(n + 1) \otimes b = b \otimes (n + 1)			       
        \]
        \[
			nb + b = bn + b        
        \]
        \[
			nb + b = bn + b        
        \]
        \[
			nb + b - b = bn + b - b        
        \]
        \[
			nb = bn       
        \]
        \item{$a \otimes (b \otimes c)=(a\otimes b)\otimes c$}
        Caso base: c = 1
        \[
			  a \otimes (b \otimes 1) = (a \otimes b) \otimes 1      
        \]
        \[
			  a \otimes b = a \otimes b 
        \]
        Hi Inductiva: c = n + 1
        \[
			a \otimes (b \otimes (n + 1)) = (a \otimes b) \otimes (n + 1)		        
        \]
        \[
			a \otimes (bn + b)) = abn + ab		        
        \]
        \[
			abn + ab - ab = abn + ab - ab		        
        \]
        \[
			abn = abn		        
        \]
        \item{$(a\oplus b)\otimes c = (a\otimes c) \oplus (b \otimes c)$}\\
        Hi inductiva
        \[
			((a + b) \otimes (n + 1)) = (a \otimes (n + 1) + (b \otimes (n + 1)))        
        \]
        \[
			an + b + a + bn = an + a + bn + b        
        \]
        \[
			(an + bn) + a + b = an + bn + a + b
        \]
        \[
			n(a + b) + a + b = n(a + b) + a + b			
        \]
        \[
			n(a + b) + a + b - a - b = n(a + b) + a + b - a - b			
        \]
        \[
			n(a + b) = n(a + b)			
        \]
\end{itemize}



\end{document}